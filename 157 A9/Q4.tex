\documentclass{article}
\usepackage[left=2cm,right=2cm,top=3cm,bottom=3cm]{geometry}
\usepackage{mathtools}
\usepackage{amsmath}
\usepackage{bbold}
\newcommand{\mprime}{m_{i}^{\prime}}
\newcommand{\Mprime}{M_{i}^{\prime}}
\newcommand{\M}{M_{i}}
\newcommand{\m}{m_{i}}
\begin{document}
\subsection*{a.)}
Suppose $M_{i} \geq 0, m_{i} \geq 0$:\\
Then, $M_{i} = M_{i}^{\prime}, m_{i} = m_{i}^{\prime}$, so $M_{i}^{\prime} -  \mprime = \M - \m$\\

Suppose $M_{i} \geq 0, m_{i} \leq 0$:\\
Then $0\leq m_{i}^{\prime} \leq |m_{i}|$ and $M_{i}^{\prime} = \text{max}\{M_{i}, |m_{i}|\}$. But, $\Mprime - \mprime  \leq \Mprime - |\m| = \text{max}\{\M - |\m|, 0\} \leq \M$. So, $\Mprime - \mprime \leq \M \leq \M - \m$.\\

Finally, $\M \leq 0, \m \leq 0$:\\
Then, $\Mprime = |\m|, \mprime = |\M|$, and $\M - \m = |\m| + \M = |\m| - |\M| = \mprime - \Mprime.$\\

So, in all cases, $\Mprime - \mprime \leq \M - \m$.
Then, from the definitions of $U(f, \mathcal{P}) \text{ and } L(f, mathcal{P})$ and the corresponding upper and lower sums on $|f|$, we have 
\begin{align*}
\sum_{i=1}^{n}\Mprime(t_{i-1} - t_{i}) - \sum_{i=1}^{n}\mprime(t_{i-1} - t_{i}) &= \sum_{i=1}^{n}(\Mprime - \mprime)(t_{i-1} - t_{i})\\
&\leq \sum_{i=1}^{n}(\M - \m)(t_{i-1} - t_{i})\\
&= \sum_{i=1}^{n}\M(t_{i-1} - t_{i}) - \sum_{i=1}^{n}\m(t_{i-1} - t_{i})\\
&=U(f, \mathcal{P}) - L(f, \mathcal{P})
\end{align*}
\subsection*{b.)}
If $f$ integrable, $\exists \mathcal{P}: U(f, \mathcal{P})- L(f, \mathcal{P}) < \varepsilon$. Since from the above, $U(|f|, \mathcal{P}) - L(|f|, \mathcal{P}) \leq U(f, \mathcal{P}) - L(f, \mathcal{P}) < \varepsilon$, $|f|$ satisfies the Riemann criterion, so it is integrable.
\subsection*{c.)}
$f + |f| = 2f_{+}$, so $\frac{f}{2} + \frac{|f|}{2} = f_{+}$. Since we've proven that both of those functions are integrable, so is $f_{+}$. The same arugment applies to $f_{-}$, which can be written as $\frac{f}{2} - \frac{|f|}{2}$.
\end{document}

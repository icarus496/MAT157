\documentclass{article}
\usepackage[left=2cm,right=2cm,top=3cm,bottom=3cm]{geometry}
\usepackage{mathtools}
\usepackage{amsmath}
\usepackage{bbold}
\begin{document}
On any interval [a,b] in [0,1], there is an irrational number, since the irrationals are dense. So, on any interval, the infimum of $f$ is 0, so all lower sums are 0. In that case, to satisfy the Riemann Criterion, it suffices to show that for all $\varepsilon > 0$ we can find a partition such that $U(f, \mathcal{P}) < \varepsilon$. We know that the max value of $f$ on [0, 1] = 1, since no fraction in lowest terms will have a denominator less than 1, and 1 is a rational number $\frac{1}{1}$. The same argument can be applied to the first interval, [0, $t_1$], since $0 = \frac{0}{1}$. It follows, then, that $M_{1} = M_{n} = 1$. Similarly, on (0, 1), the max value of $f$ is $\frac{1}{2}$. 
 
\end{document}

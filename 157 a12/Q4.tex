\documentclass{article}
\usepackage[left=2cm,right=2cm,top=3cm,bottom=3cm]{geometry}
\usepackage{mathtools}
\usepackage{amsmath}
\usepackage{bbold}
\newcommand{\sfunc}{S_{\mathcal{P}}}
\newcommand{\si}{sin\lambda x}
\begin{document}
\subsubsection*{4a}
\begin{align*}
	\int_{a}^{b}sin(\lambda x) dx &= -\frac{cos(\lambda x}{\lambda}|^{b}_{a}\\
	&= \frac{-cos \lambda b}{\lambda} + \frac{cos \lambda a}{\lambda}\\
	\lim_{\lambda \rightarrow \infty} \int_{a}^{b}sin(\lambda x) dx &=  \lim_{\lambda \rightarrow \infty} (\frac{-cos \lambda b}{\lambda} - \frac{cos \lambda a}{\lambda})\\
	&=\lim_{\lambda \rightarrow \infty} \frac{1}{\lambda} (cos \lambda a - cos \lambda b)\\
	&\text{Since $-1 \leq cos \lambda x \leq 1$}\\
	\frac{-2}{\lambda} \leq \frac{1}{\lambda}(cos \lambda a &- cos \lambda b) \leq \frac{-2}{\lambda}\\
	\lim_{\lambda \rightarrow \infty} \frac{-2}{\lambda} &= \lim_{\lambda \rightarrow \infty} \frac{2}{\lambda}\\ 
\end{align*}
So by the squeeze theorem, $\lim{\lambda \rightarrow \infty} \frac{1}{\lambda} (cos \lambda a - cos \lambda b) = \lim_{\lambda \rightarrow \infty} \int_{a}^{b}sin(\lambda x) dx = 0$.
\subsubsection*{4b}
\begin{align*}
lim_{\lambda \rightarrow \infty} \int_{a}^{b}s(x)sin\lambda x dx &= lim_{\lambda \rightarrow \infty} \sum_{i=1}^{n} \int_{t_{i-1}}^{t_{i}}s(x)sin\lambda x dx\\
&= lim_{\lambda \rightarrow \infty} \sum_{i=1}^{n} \int_{t_{i-1}}^{t_{i}}c_{i}sin\lambda x dx  && \textit{Since $s$ is a step function on $[t_{i-1}, t_{i}]$}\\
&= \sum_{i=1}^{n} c_{i} lim_{\lambda \rightarrow \infty}  \int_{t_{i-1}}^{t_{i}}sin\lambda x dx\\
&= \sum_{i=1}^{n} c_{i}(0)\\
&= 0
\end{align*}
\subsubsection*{4c}
Let $S_{\mathcal{P}} = (x \in [t_{i-1}, t_{i}]: m_{i})$. Forgive the abuse of notation but it's just a step function defined on the partition $\mathcal{P}$ equal to the infimum of $f$ on each interval in the partition. Then, $\sfunc$ is integrable, since $U(\sfunc, \mathcal{P}) = L(\sfunc, \mathcal{P})$. So, we know:
\begin{equation*}
\forall \varepsilon > 0, \exists \mathcal{P}_{1}: U(S_{\mathcal{P}_{1}}, \mathcal{P}_{1}) - \int_{a}^{b} S_{\mathcal{P}_{1}} < \varepsilon\\
\end{equation*}
\begin{equation*}
\forall \varepsilon > 0, \exists \mathcal{P}_{2}: U(f, \mathcal{P}_{2}) - \int_{a}^{b}f < \varepsilon\\
\end{equation*}
\begin{equation*}
\forall \varepsilon > 0, \exists \mathcal{P}_{3}: U(f, \mathcal{P}_{3}) - U(S_{\mathcal{P}_{3}}) < \varepsilon\\
\end{equation*}
We can rearrange these to get 
\begin{equation*}
\forall \varepsilon, \exists \mathcal{P}: \int_{a}^{b} f - \int_{a}^{b} \sfunc < \varepsilon
\end{equation*}
Now, consider the equation:
\begin{align*}
\left| \int_{a}^{b} f\si - \int_{a}^{b} \sfunc \si \right| &= \left|\int_{a}^{b}(f - \sfunc) \si \right|\\
&< \left|\int_{a}^{b} \varepsilon \si \right|\\
&\leq \int_{a}^{b} \varepsilon |\si|\\
&\leq \int_{a}^{b} \varepsilon\\
&= (b-a) \varepsilon\\
&= \varepsilon &&\text{Since $\varepsilon$ is arbitrary}
\end{align*}
So, we know that for all $\varepsilon$ there exists a $\mathcal{P}$ such that
\begin{align*}
\left| \int_{a}^{b} f\si - \int_{a}^{b} \sfunc \si \right| &< \varepsilon\\
lim_{\lambda \rightarrow \infty} \left| \int_{a}^{b} f\si - \int_{a}^{b} \sfunc \si \right| &< lim_{\lambda \rightarrow \infty} \varepsilon\\
\left| lim_{\lambda \rightarrow \infty}  \int_{a}^{b} f\si  - 0 \right| &< \varepsilon\\
-\varepsilon \leq  lim_{\lambda \rightarrow \infty} & \int_{a}^{b} f\si \leq \varepsilon
\end{align*}
Since this is true for all $\varepsilon$, we know that the limit must be 0.
\end{document}
